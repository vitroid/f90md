\documentclass[a4,10pt]{article}
\usepackage{bxascmac}
\usepackage[a4paper, portrait, margin=2cm]{geometry}
\parskip7pt
\parindent0pt
\usepackage{setspace}
%\doublespacing
\onehalfspacing

%% Utilities for TextWrangler
%% ([A-Za-z_0-9.\\\[\]]{2,})
%% {\\tt \1}
\title{f90md -- Molecular Dynamics in fortran90}
\begin{document}
\maketitle
%------------------------------------------------------------
\section{000Introduction}
%------------------------------------------------------------
This is a short tutorial to make a molecular dynamics code in fortran90.

%最速なプログラムを書くのにはあまり役に立ちません。複雑な分子を扱うことも今のところできません。
%しかし、プログラムの書きはじめ方、コンパイルの仕方、プログラムの分割の方法、
%機能を追加する方法などの参考にはなるかと思います。


\input git.tex
%------------------------------------------------------------
\section{010TwoBody}
%------------------------------------------------------------
Let us suppose a system of two particles connected by a spring.
Their action can be estimated by solving the following equation of motion numerically.

%分子動力学法の最も単純な例として、質量1の2つの質点がバネ定数1のバネでつながれた系の
%運動(座標の時間変化)を、運動方程式を数値的に解くことで求めます。

%物体の運動は、ニュートンの運動方程式
\begin{equation}
  {\bf F} = m {\bf a},
\end{equation}
where ${\bf F}$ is the force vector to a particle, m is the mass of
the particle, and ${\bf a}$ is the acceleration vector.
%に従います。ここで、Fはバネが物体に及ぼす力、mは質量、aは加速度です。
%バネの力は、2つの物体の相対座標で決まります。また、加速度は座標の時間での2回微分ですから、この式を物体のかたほうについてもっと具体的に書くと、
The force comes from the spring connecting two particles.  Here we
assume that the strength of the force is proportional to its length, i.e., 

\begin{equation}\label{eq:f_sp}
{\bf F}_1 = -k ({\bf r}_1- {\bf r}_2) = m_1 {\bf a}_1=-m_2 {\bf a}_2,
\end{equation}
%と書けます。力$\bf F$と位置$\bf r$と加速度$\bf a$はベクトルです。$k$はバネ定数です.作用反作用の法則により,もう一方の末端には逆向きの力が加わります.
where the force $\bf F$, position $\bf r$, and the acceleration $\bf
a$ of a particle are vectors, and $k$ is the spring coefficient. 
The second equivalence in the equation indicates the Newton's third
law, i.e. while one particle is pulled, the other
particle is pushed by the force with the same strength but opposite direction.

These equations can also be derived from the potential energy function of a spring by differentiation.
\begin{eqnarray}
V({\bf r}_1, {\bf r}_2) &=& \frac{k}{2}\left({\bf r}_1-{\bf
    r}_2\right)^2\\
{\bf F}_1&=& -\frac{\partial V}{\partial{\bf r}_1}\\
&=&-k\left({\bf r}_1-{\bf r}_2\right)\\
{\bf F}_2&=& -\frac{\partial V}{\partial{\bf r}_2}\\
&=&-{\bf F}_1
\end{eqnarray}

%ところで、加速度は速度を時間で微分したもの
Acceleration is a derivative of the velocity.
\begin{equation}
a_x = \partial v_x / \partial t
\end{equation}
%また速度は座標を時間で微分したものです。
Velocity is also a derivative of the position,
\begin{equation}
v_x = \partial r_x / \partial t
\end{equation}
where $_x$ indicates the $x$ component.

By introducing finitely small time interval $\Delta t$, these
differential equations can be rewritten as a finite differential
equations.
\begin{eqnarray}
a(t)&=&[ v(t+\Delta t) - v(t) ] / \Delta t\\
v(t)&=&[ r(t+\Delta t) - r(t) ] / \Delta t
\end{eqnarray}
It can also be written in the following form.
\begin{eqnarray}
v(t+\Delta t) = v(t) + a(t) \Delta t\\
r(t+\Delta t) = r(t) + v(t) \Delta t
\end{eqnarray}
That is, if we know the positions and velocities of the particles at
time $t$, firstly we can calculate the force between the particles by
eq.\ref{eq:f_sp} and obtain the accelertions, then we can estimate the
velocities and positions of the particles at near future by these
finite differential equations.
%と書けます。つまり、現在の座標から力を計算し、力から運動方程式によって現在の加速度を計算し、現在の速度と加速度から少し未来の速度を、そして現在の座標と速度から少し未来の座標を計算できることがわかります。

This is the core part of the program.  Refer the program {\tt
  main.f90} and look for the equations.  Consider how these equations
are implimented as a fortran90 code.
%この計算を繰り返し行うことで、物体の未来の位置と速度を計算できます。これが分子動力学法の中核です。プログラム{\tt main.f90}を上の式と見比べて、どこでどの計算をやっているかを想像してみて下さい。

%Well, if 
%このプログラムを実行するには、コンピュータが理解できる形に「翻訳」す
%る必要があります。この作業をコンパイルと呼びます。{\tt main.f90}をコ
%ンパイルして、実行ファイル{\tt main}を作るには、ターミナルで
When you write a program, you have to translate the human-readable
programming language to the machine-native code.  Thi s process is
called ``compilation''.  Type the following line in the terminal
(Terminal.app in MacOS) to compile {\tt main.f90} into a
machine-executable {\tt main} file.
\begin{screen}
\begin{verbatim}
gfortran main.f90 -o main
\end{verbatim}
\end{screen}
%と入力します。

%エラーなくコンパイルできたら、
When the code gets compiled without errors, let's run the code by
typing the following command in the terminal.
\begin{screen}
\begin{verbatim}
./main
\end{verbatim}
\end{screen}
%と入力して下さい。
%{\tt "./"}は、``現在のディレクトリにある、''という意味を表しています。画面には何か数字がたくさん表示され、いかにもたくさん計算している雰囲気になります。
{\tt "./" } specify that the location of the file is in the current
working directory.  You will see a lot of numbers in the terminal.
Enjoy the feeling of mass calculation.

%ちゃんとプログラムが動いていて、物理法則にのっとった運動をしているかどうかを確認するには、エネルギー保存則をチェックするのが簡単です。この系の場合、ポテンシャルエネルギーはバネのエネルギー
Is the program running correctly?  One easy way is to draw the
trajectory of the particles in the graph using GNUPLOT.

Another numerical check can be done by the law of energy
conservation.  According to the law, total energy (i.e. sum of the
kinetic and potential energies) must be conserved all time.

In this system, potential energy is the energy of the spring.
\begin{equation}\label{eq:e_sp}
E_p = k \left| r_1 - r_2 \right|^2 / 2
\end{equation}
Kinetic energy can be calculated for each particle.
%運動エネルギーは2つの物体それぞれ
\begin{equation}
E_k = m \left| v \right|^2 / 2
\end{equation}
In fact, the program is showing these values on the screen.
%です。プログラムでは、これらの値を画面に出力するようになっています。

The total energy is not truly constant but fluctuates, because we
approximate the differential equations with the finite differential
ones.
Still, you can check that the fluctuation of the total energy is much
smaller than the fluctuation of the kinetic or potential energy
(e.g. by plotting them with {\tt gnuplot}).
%数値計算ですから、全エネルギーが厳密に一定になるわけではありませんが、運動エネルギーやポテンシャルエネルギーの時間変動に比べると、それらの和はほとんど変動しないことを、{\tt gnuplot}などでプロットして確認して下さい。

\begin{itembox}[l]{Practice 1}
In general, when the potential energy is the function of
the positions of the particles, you can obtain the force applied to
each  particles by differentiating the function by the position of the
particle.

In the case of two particles connected by a spring, you get the force
(\ref{eq:f_sp}) by differentiating the potential energy of the spring
(\ref{eq:e_sp}) by the particle position $r_A$.
\begin{eqnarray}
E_p(r)&=&kr^2/2\\
{\bf F}&=&-\frac{\partial E_p(r)}{\partial {\bf r}}\\
&=&-\frac{\partial E_p(r)}{\partial r}\frac{\partial r}{\partial {\bf r}}\\
&=&-k{\bf r}
\end{eqnarray}
Differentiate the Coulomb interaction function $V(r) = k / r$ to
obtain the formula of the coulombic force vector.
\end{itembox}

\begin{itembox}[l]{Practice 2}
Write a small program to validate the result of the practice 1 by finite differentiation,
i.e. calculate the potential energy difference when a particle is
moved at small amount to the axis directions.  The rate of change must
be close to the components of the force vectors obtained as the derivative.
\end{itembox}

%------------------------------------------------------------
\section{015Displaced}
%------------------------------------------------------------

The equations used in the previous section assume that the equilibrium
(natural) distance between the two particles is zero.  However, the
equilibrium length of the real spring is not zero but a finite value.
So let us modify the formula for the more realistic spring.  This idea
is also useful when the intramolecular covalent bond is represented
with a spring with a finite natural length.

The interaction potential becomes minimum when the two particles are in its natural length $L$.
\begin{equation}
V({\bf r}_1, {\bf r}_2) = \frac{k}{2}\left(\left|{\bf r}_1-{\bf r}_2\right| - L\right)^2.
\end{equation}

You get the force by differentiating the equation by the positions of
the particles in the same way as the previous section.
\begin{eqnarray}
{\bf F}_1&=& -\frac{\partial V}{\partial{\bf r}_1}\\
{\bf F}_2&=& -\frac{\partial V}{\partial{\bf r}_2}\\
\end{eqnarray}

\begin{itembox}[l]{Practice 2.5}
Derive the equation of the force by performing differentiations above.
\end{itembox}


%------------------------------------------------------------
\section{020TwoBodyLJ}
%------------------------------------------------------------

Let us implement the Lennard-Jones interaction derived in Practice 1 in place of the spring interaction.  Now the particles are unbound.  The force and energy calculation becomes a little bit complicate.

\begin{itembox}[l]{Practice 3}
Write a small program to validate the result of Lennard-Jones interactions in the same way as Practice 3.
\begin{equation}
    E_p(r) = A/(r^p) - B/(r^q)
\end{equation}
\end{itembox}


%------------------------------------------------------------
\section{030NBody}
%------------------------------------------------------------

Let us increase the number of particles from two to eight.
Now it becomes hard to write all the combination of the pairwise interactions.
Therefore, some loops are added in the program.
The number of particles will become variable in the later stage.
%質点の個数を8個に増やします。
%座標を個々に指定するのは面倒なので、格子点の座標を生成するループを追加しました。
%今のところ、分子数は8個に固定ですが、あとで自由に増やせるようにします。


%------------------------------------------------------------
\section{040Readability}
%------------------------------------------------------------

As the program is getting longer and longer, it also becomes harder to understand.
Let us have a break now.  Instead of adding new features, let us make the program easier to read.

First improvement is by giving  some descriptive names to the variables.
It is better to name {\tt position} instead of {\tt x,y,z} for position, and {\tt veolcity} instead of {\tt vx,vy,vz} for velocity.
The use of longer variable name does not have side effects but reduces the mistakes.
For example, when you write a formula for {\tt vx} and copy it to make the formulae for {\tt vy} and {\tt vz}, you often leave some ``vx''s unmodified by mistake.  Since {\tt vx} is a correct variable name, the compiler does not warn you.

Another improvements is the use of multi-dimensional arrays.  Let us use 3-element array instead of making {\tt x()}, {\tt y()}, and {\tt z()} arrays.  For example, the positions of the particles should not be written like
\begin{screen}\begin{verbatim}
real(kind=8) :: x(8),y(8),z(8)
\end{verbatim}\end{screen}
but be like
\begin{screen}\begin{verbatim}
real(kind=8) :: position(3,8)
\end{verbatim}\end{screen}

In {\tt fortran90}, there is a smart syntax to apply the same calculations to all the array elements.  For example, position arrays will be initialized in the traditional fortran syntax like:
\begin{screen}\begin{verbatim}
 do i=1,8
   x(i) = 0d0
   y(i) = 0d0
   z(i) = 0d0
 end do
\end{verbatim}\end{screen}
but in fortran90, you can write simply as
\begin{screen}\begin{verbatim}
 position(:,:) = 0d0
\end{verbatim}\end{screen}

This syntax reduces the number of lines and improve the readability.  Moreover, this is easier for the compiler to understand what the programmers want to do, and that enables the compiler to produce the faster codes.

\begin{shadebox}
Let us write a readable program and put many comments in it.  You of the future will be the person who read your program most frequenty, and you of the future is a different person from you of now.  Better coding also makes you easier to modify and add new features to the program.  %    プログラムを読みやすく書く習慣をつけましょう。他人に見せないからスタイルは気にしない、と言う人もいますが、スタイルを気にせず書いてしまったから他人には見せたくても見せられないのかもしれません。きれいなプログラムを書くのは、他人のためではなく、未来の自分のためです。自分が書いたプログラムを一番頻繁に読むのは、自分自身です。どんなに時間をかけて書いたプログラムでも、しばらく時間がたつと全体像があやふやになります。分かりやすいプログラムを書いておけば、次にプログラムを改良する必要が生じたときに、間違いをおかさず、すみやかに作業を終えることができます。汚なく書いてもかまわないのは、使い捨ての短いプログラムだけです。
\end{shadebox} 

\begin{shadebox}
Make the backups before making large changes to the programs.  Compare the results of before and after the modifications, and make sure that the results are the same for the same input.  It is nice to use an automatic backup tool (e.g. TimeMachine on MacOS).
\end{shadebox}



%------------------------------------------------------------
\section{050FileIO}
%------------------------------------------------------------

It is more comvenient to read the initial positions of the molecules from a file. %プログラムの中で分子の座標を生成するだけだと、任意の初期配置から計算をはじめることが
%できないので、座標はファイルから読みこめるようにしましょう。
In the programs in the previous sections, the number of molecules in a system is known in advance, and all the variables are prepared as to be the predetermined sizes.  In case the molecules are read from a file, the number of molecules are unknown in advance, and so is the size of the arrays.

%ファイルから読みこむ場合、読みこみはじめるまで、分子の個数がわかりませんから、
%配列もどれだけの大きさにしておけばいいかわかりません。
%1000分子分の配列をとってあっても、実際に読みこんだ分子数が10個ならメモリの無駄遣い
%(メモリを無駄に使うと、処理速度は落ちます)だし、逆に10個分しか配列を作っていないのに
%1000分子を読みこもうとすると、エラーになります。

We therefore introduce an ``allocatable'' array whose size is decided afterwards (i.e. not in the compilation time but in the run time).
%そこで、配列の大きさをあとから指定できるように、座標の配列を可変長としておきます。
This makes {\tt position} array as allocatable.
\begin{screen}\begin{verbatim}
real(kind=8), allocatable :: position(:,:)
\end{verbatim}\end{screen}

We here assume that the input file starts from a single line describing the number of molecules, and the following lines contains the positions of the molecules.
%読みこむ座標ファイルには、最初の行に分子数が書いてあり、
%そのあとに座標が分子数分列挙されているものとします。
%ファイルの最初の行を読みこんだら、その値に応じて配列の大きさを決定します。
The program reads the first line and allocate the size of the {\tt position} array.
\begin{screen}\begin{verbatim}
allocate(position(3,num_molecule))
\end{verbatim}\end{screen}

In this branch, the program to prepare the input file is also included.  The program, {\tt scl.f90}, prepares a 3x3x3 lattice of simple cubic lattice whose unit lattice size is $4 {\rm \AA}$, and output it to the screen.

The MD program, {\tt main.90}, reads this as an input.  The way to combine these programs is written in {\tt run.sh}.
The shell script {\tt run.sh} makes the initial positions of the molecules and then run the molecular dynamics simulations.


%------------------------------------------------------------
\section{060Unit}
%------------------------------------------------------------

By the way, what units are we using in this program?  It is not clearly mentioned in this document.

In fact, we can choose any combination of units as far as they are consistent.  In this document, we are using the following units.
\begin{center}
\begin{tabular}{cc}
\hline
Energy&kJ/mol\\
Pressure&Pa\\
Length&Angstrom = $10^{-10}$ m\\
Time&pico sec = $10^{-12}$ sec\\
Mass&?\\
 \hline
 \end{tabular}
\end{center}
\begin{itembox}[l]{Practice 4}
Assume the mass unit used in the program.
\end{itembox}
%------------------------------------------------------------
\section{070TaggedFileIO}
%------------------------------------------------------------
In the previous section, the number of molecules and the molecular positions are input from the file.  While, number of time steps, molecular mass, and other configurable values are written directly in the program.

It is more convenient if all these configurable values are in the input file because we need not modify the program for the purposes.

We must be conscious about the roles of the program, input data, and output data.  They are strongly related to each other.  We prepare various input files to simulate various different conditions, and they make different output files (results).  In this sense, we use the same program for different inputs and outputs.  While, we also change the program to extend it when we deal with different systems.  In other words, program evolves.

It is therefore very important to make the input and output data files ``self-descriptive''.  In the previous section, we prepared an input file made only of the numbers.  Suppose if you change the program to read the pressure from the input file.  Then, the input file requires one more number to specify the pressure.  The old input file without the pressure value is no longer useful with the new program, and you lose the way to ``percept'' the meaning of the values on the old input file.  Thus the program evolves day after day, and if the input and output data are made only of numbers, you will easily lose the meanings of the values in the files and the file becomes a garbage.

In order to avoid the loss of the meanings, it is strongly recommended to make the input and output files ``self-descriptive'', i.e. you should add additional information (or tags) describing the meanings of the values.  They are not only for the compatibility of the data and program, but also for the human who uses the data and extends the program.

You need not be very conscious about the way to describe the meaning of the values.  The standard way for data tagging is XML (eXtensible Markup Language), but it might be too strict for us.  Feel free to use your own tagging syntax, because it is easy to convert the tagged data from one format to another.  

For example, we tag the number of atoms with the bracketed tag followed by an integer.
\begin{screen}\begin{verbatim}
[Number Of Atoms]
27
\end{verbatim}\end{screen}
All right, but the tag may be too long. We may misspell it and that may cause another trouble.  So we shorten the tag like this.
\begin{screen}\begin{verbatim}
[NATOM]
27
\end{verbatim}\end{screen}

It must be noted that when you change the program and change the meaning of the value, you must also change the tag.  For example, do not use ``NATOM'' tag for the number of molecules.  Appropriation of the tags will badly introduce confusions.

It must also be warned that the tags should not be ``too descriptive''.  For example, suppose you introduce a tag to specify the positions of atoms for the molecular dynamics simulation of argon,
\begin{screen}\begin{verbatim}
[ARGON POSITIONS]
\end{verbatim}\end{screen}
and prepare several analysis program to read the file with this data format.

After a while, you might want to do the similar simulation with xenon atoms.  if you introduce another tag specific to xenon like this,
\begin{screen}\begin{verbatim}
[XENON POSITIONS]
\end{verbatim}\end{screen}
You must change all the analysis programs to accept the xenon data.  This is the case for the ``too descriptive'' tag.  It is better to share the same (less descriptive) tag like
\begin{screen}\begin{verbatim}
[POS3]
\end{verbatim}\end{screen}
that enables you to share the same analysis programs for different atom species.

In the programs, any configurable values such as Lennard-Jones interaction parameters, loop counts, $\Delta t$ for integration, etc. are tagged.

The program to make the initial configuration, {\tt scl.f90}, is also modified to include the tags.

%------------------------------------------------------------
\section{080Cell}
%------------------------------------------------------------

Let us introduce the simulation box (we call it ``a cell'') with periodic boundary (left, top, and front sides of the box are virtually connected to the right, bottom, and back sides, respectively.)

Very small amount of changes are made to introduce the periodic boundary condition.  Read the program carefully to understand the technique.


%------------------------------------------------------------
\section{090Pressure}
%------------------------------------------------------------

Once the volume is determined, we can calculate the pressure.  In the molecular dynamic simulation, pressure is calculated from virial.


%------------------------------------------------------------
\section{100Modular}
%------------------------------------------------------------

The program is getting longer an longer.  Even if you put a lot of comments in the program, it gets harder to understand the program.  A single unit of a program should be about 50 lines for easy understanding.

In this section, we divide the program into the modules.  This makes functions of each module simpler, and lets you understand it easier.
%こんなこと書くぐらいなら,moduleはやめたほうがいいなあ.
\subsection{Some notes on \tt module}
\subsubsection{{\tt module} is neither a structure or an object}
In programming language, a structure means a package of data, and an object means a package of data and codes.  A module also contains some data and codes, but it is a different concept from structures and objects.  In C language, for example, you can define a structure and make an array of the structure type.  In this sense, a structure is copiable dynamic package of data.  However, you cannot make a copy of the fortran module.  The module is just a static, non-copiable set of codes and data.  I think the module in fortran90 is introduced just to avoid the use of  ``common'' syntax, which caused many serious bugs and bad customs in Fortran77 codings.  Now modular programming is not popular at all, while object-oriented programming style is common in the popular programming languages.  The structure is also available in fortran90, and the object-oriented programming is available in fortran95 and later.  I therefore do not recommend you the deeper use of modules in fortran90.
\begin{quote}
If all you have is a hammer, everything looks like a nail.
\end{quote}


\begin{itemize}
%\item {\tt module}内変数はグローバル変数。
\item モジュール間での名前の衝突
\item モジュールに分けると読みやすくなるが、安全にはならない。
\item {\tt intent}を使おう。
\item あまり依存関係が複雑になりすぎないようにする。
\item 関数名のつけかた
\item (分割コンパイルの方法。)
\end{itemize}



%------------------------------------------------------------
\section{110Symplectic}
%------------------------------------------------------------

Let us reconsider the numerical integration.  The program is now separated into several modules and we can easily modify it.

%------------------------------------------------------------
\section{Extension to the multicomponent system}
%------------------------------------------------------------

混合物を扱うために、どんな拡張が必要になるか考えてみましょう。
とりあえず、分子はこれまでと同じく、単原子分子を想定します。

まず、2種類以上の分子種を扱うわけですから、分子の座標や速度の情報も、分子種ごとに保持する必要があります。

一つの方法としては、{\tt properties}モジュールに含まれる、座標や速度の配列を長くして、
全分子をおさめてしまうという手が考えられます。
しかし、これだと、どこまでがどの分子の情報かわかりにくいし、
おそらく分子種が増えてくるにつれて、非常に読みづらいプログラムになると思われます。

では、{\tt properties}モジュールを複製して、もうひとつ{\tt properties2}モジュールを作ったらどうでしょうか。
確かに、2種類の分子を、2種類のモジュールで管理すれば、先の場合のようなややこしさは回避できます。
しかし、3種類、4種類と分子種が増えてくるにつれて、モジュールを増やしていくと、ほぼ同じ内容の
モジュールがいくつもでき、メンテナンス上の問題がだんだん無視できなくなります。
\footnote{モジュールの配列みたいなものが作れればいいんですけど,先にも書いたようにモジュールはオブジェクトとは違うので,1つ作って複製する,というわけにいきません.}

では、一つの{\tt properties}モジュールの中で、座標や配列の次元を1つ増やして、分子種を使いわけるという方法ではどうでしょう。これは一見うまくいきそうですが、例えば100万分子の成分Aの中に、1分子の成分Bが溶けているような状況を考えた場合、成分Bの配列はほとんど無駄になります。これも、スマートな解決法とは言えません。

もう一つの方法として、{\tt properties}内のモジュール変数を、全部構造体にまとめてしまい、その構造体を、分子種ごとにいくつも動的に作成する、という方法があります。この方法なら、構造体ごとにメモリ使用量を調節できますし、分子種がいくつ増えてもその都度モジュールを増やす必要はありません。以下では、この方針で、プログラムを書換えていくことにします。

%------------------------------------------------------------
\section{130Mixture}
%------------------------------------------------------------

さて、{\tt properties}モジュールには、分子の座標や速度、個数、質量、Lennard-Jones相互作用パラメータなどの情報が含まれています。このうち、Lennard-Jones相互作用パラメータは本当に分子の属性と呼べるでしょうか。というのも、2種類以上の分子が混在する場合、相互作用パラメータは、分子種の個数分だけでなく、異種分子の組みあわせすべてについて、個別に適切に設定する必要があります。通常は、異種Lennard-Jones分子間の相互作用はLorentz-Berthelot則を適用し、
\begin{equation}
 \sigma_{12} = (\sigma_1+\sigma_2)/2,\\
 \epsilon_{12} = \sqrt{\epsilon_1 + \epsilon_2}
 \end{equation}
で計算しますが、それをプログラムの中にハードコードしてしまうのはあまりスマートとは言えません。分子間相互作用のパラメータは、分子の属性とするよりも、力計算モジュールの属性とするほうが自然でしょう。そこで、まず最初のステップとして、Lennard-Jones相互作用パラメータを、{\tt force\_module}に移動し、モジュールの名前も{\tt interaction\_module}に変更します。(次のセクションでさらに大改造します)

%------------------------------------------------------------
\section{135Mixture2}
%------------------------------------------------------------

構造体を導入するには、さらに大手術が必要になります。まず、{\tt properties\_module}は、いろんな種類の分子をまとめる場所として使い、個々の分子団は別のモジュールにわけます。ここでは、単原子分子のつもりで、{\tt monatom\_module}を作りました。単原子分子の構造体はこのモジュールで定義し、この構造体を操作するサブルーチンも全部このモジュールの中におしこみます。

{\tt monatom\_module}の中身は、旧{\tt properties\_module}とほとんど同じですが、後者ではモジュール変数に対して処理を行っていたのを、前者では構造体に対して処理を行うように書換えました。これにより、同じサブルーチンを、異なる分子群に対して使えるようになります。

{\tt properties\_module}のほうは、混合物の成分数だけ{\tt monatom\_type}構造体を持ち、個々のサブルーチンが{\tt main}から呼びだされると、各成分に処理をふりわけるだけの役割になります。

\begin{shadebox}
    このチュートリアルでは、モジュール名には必ず{\tt \_module}を、構造体名には必ず{\tt \_type}を    付けることにしています。このように、名前がデータの型を表すようにするのは、少し冗長な感じもしますが、プログラムが大きくなってきて、全体が見渡せなくなってきたり、多人数で開発を行うようになった時に、読み間違い/書き間違いが生じないようにする工夫です。もっと学びたい人は「変数の命名規則」で検索してみて下さい。
\end{shadebox}
多成分系では、データの読み込み方法も工夫が必要になります。これまでなんとなくArgon分子のパラメータとしてきた、座標や速度、質量、相互作用パラメータのうち、相互作用パラメータだけは、扱いが大きくかわります。

というのも、相互作用は、その名の通り、分子そのものの属性というよりは、分子(群)の間で定義されるものだからです。2成分系であれば、座標、質量は2種類の分子の情報を読みこむ必要がありますが、相互作用は、3種類(同種同士が2通り、異種分子間相互作用が1通り)を定義する必要があります。

まず、分子の情報である、座標と分子量を読み込む前に、[{\tt COMPONENT}]というタグを付けて、新しい成分についてのデータがこのあとに来るよ、ということを明示するようにしました。
[{\tt COMPONENT}]が入力データに出現するたびに、成分が1つずつ増えます。

そして、相互作用は{\tt [COMPONENT]}とは全く別に、{\tt [INTRPAIR]}タグで指定することにします。これまでの{\tt [LJPARAM]}は使わなくなり、将来の拡張性も見据えて、相互作用の種類を指定できるようにします。

例えば、成分1と成分1がLennard-Jones相互作用する場合には、
\begin{screen}\begin{verbatim}
 [INTRPAIR]
 1 1 LJ
 <LJのepsilon> <LJのsigma>
\end{verbatim}\end{screen}
という形で指定します。将来、Lennard-Jones以外の相互作用を扱う場合には、"LJ"の部分が別のラベルに代わり、3行目のパラメータもそれに応じて個数が変わってきます。

"LJ"の代わりに"LB"と書くと、Lorentz-Berthelot則でepsilonとsigmaを計算します。
この場合は3行目のパラメータは不要です。
\begin{screen}\begin{verbatim}
 [INTRPAIR]
 1 2 LB
\end{verbatim}\end{screen}

ところで、
\begin{screen}\begin{verbatim}
 1 1 LJ <LJのepsilon> <LJのsigma>
\end{verbatim}\end{screen}
のように全部1行に書いてしまえばいいじゃないかと思うかもしれません。しかし、上のように分けておくことで、まず相互作用の種類を読みこみ、それに応じて読みこむパラメータ数を変化させることができます。fortran90でファイルからデータを読みこむ場合はいろいろ制約が多いので、データ形式もそれに配慮して定義しておくと、将来何度もデータフォーマットを変更するのを回避できます。

さて、{\tt properties}から{\tt monatom}を分離したのと同じように、{\tt interaction\_module}から{\tt lj\_module}を分離し、{\tt interaction\_module}は相互作用の処理のふりわけだけをするように書換えました。実際の相互作用の計算は{\tt lj\_module}が担当します。{\tt lj\_module}の中では、もともと{\tt force\_calculate}と呼ばれていたルーチンは、{\tt lj\_calculate\_homo}と{\tt lj\_calculate\_hetero}の2つに分かれていて、それぞれ同種分子と異種分子の間の相互作用計算を担当します。これらの振り分けは{\tt interaction\_calculate}が行います。現在のコードでは、成分数は最大100成分までとなっていますが、多少多すぎるかもしれません。{\tt lj\_pair}については、いずれ改良する予定です。

\begin{shadebox}
    {\tt lj\_module}の中のサブルーチン/関数は、必ず第一引数が{\tt lj\_type}となっています。{\tt monatom\_module}の場合も同じです。同じ構造体を扱うサブルーチンばかりをひとまとめにしたものを、プログラミングの世界ではオブジェクトと呼びます。また、そのようにまとめることで、プログラムの見通しを良くしようという考え方を、オブジェクト指向と呼びます。オブジェクト指向にすると実行速度が遅くなるといった批判もありますが、多くはJavaなどのインタプリタ言語で、動的なメモリの確保と解放を積極的に行い、なおかつ多態を多用する場合の話のようです。いまここで説明している限りでは、コンパイラ言語で、動的メモリの確保はプログラムの起動時にしかおこなわず、多態もまだ使っていないので、悪影響は心配しなくて良いと思います。このチュートリアルでは、プログラムの変更により、明らかに遅くなった場合には、何が原因かを追究し、プログラムの読みやすさを損ねずに改善する方法も紹介します。
\end{shadebox}

{\tt properties\_module}と{\tt interaction\_module}をうまく拡張したので、{\tt main}の中身はほとんど変更の
必要がありませんでした。
力の初期化は、これまでは力計算ルーチン{\tt force\_calculate}のはじめで行っていました。しかし、
多成分系では、いろんな成分の組み合わせの間で力を計算し、それらを累積する必要があります。
そこで、{\tt force\_calculate}({\tt interaction\_calculate}に改称)を呼ぶ前後で、{\tt properties\_preforce()}
を呼びだして初期化させるようにしました。

さいごに、動作試験をするための初期構造を生成する、{\tt gen\_2compo.f90}を準備しました。
{\tt scl.f90}では、単純格子を生成しました。{\tt gen\_2compo.f90}は、単純格子を2分して、2つの成分に
ふりわけます。実際には、2つの成分の相互作用は全く同じにすれば、計算結果も全く同一になるはずです。
{\tt scl.f90}のほうは、入力データ形式が変更されたのにあわせて、多少内容を変更し、{\tt gen\_scl.f90}という
名前に変更しました。

また、これらを組みあわせ、格子を生成してMDを実行するスクリプト{\tt run\_scl.sh}と{\tt run\_2compo.sh}も
準備しました。1成分でも、2成分に分割しても、全く同じ結果が得られることを確認して下さい。
また、スクリプトが生成する、{\tt main}用の入力データ{\tt *.input}の中身を見れば、
1成分の場合と2成分の場合のデータの違いを見比べることができます。

\begin{shadebox}
    シェルスクリプトなどが生成する中間ファイルは、ファイル名が"@"ではじまるように統一します。
    これは、大昔の汎用計算機時代の名残り(短期ファイルの目印)なんですが、もはや誰も覚えていないし、
    検索しても出てこないようです。田中先生なら知っているはず。
\end{shadebox}


\begin{itembox}[l]{練習問題4}
100ステップ後にファイルに出力した結果を、再度読みこんでさらに100ステップ継続計算した場合と、
    200ステップ連続で計算した場合の結果が一致するかどうかを確認するスクリプトを作りましょう。	混合物と単成分で同じ結果になるかどうかを確認して下さい。
    \end{itembox}



%------------------------------------------------------------
\section{今後の予定}

各セクションで、確認しておくべき点、テスト手法を示すか、課題とする。!!!!

\begin{itemize}
\item 実行速度の評価方法と改善方法
\item 熱浴(温度調節) Berendsen and Nose-Hoover
\item 正しい結果が得られているかどうかを検証する方法、デバッグ手法、テスト手法、分析(動径分布関数など)
\item DONE 混合物
\end{itemize}
\section{今後の未定}
\begin{itemize}
\item 剛体分子
\item カットオフ、相互作用計算の高速化
\item クーロン力、長距離相互作用の操作
\item 能勢の熱浴
\item 圧力調節
\item さまざまな境界条件
\item 並列化
\end{itemize}
\end{document}

