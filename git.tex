%------------------------------------------------------------
\subsection{Usage of git}
-%-----------------------------------------------------------
There are many variants (branches) for the programs of f90md.  The section names of this document correspond to the branches.

First of all, let us obtain all the source codes.  Open your terminal (Terminal.app on MacOS), and type the following command in some directory:
%まずは、以下のコマンドでソースツリー(すべての枝)を取得して下さい。
\begin{screen}
\begin{verbatim}
 git clone https://github.com/vitroid/f90md.git
\end{verbatim}
\end{screen}
You will get the copy of the programs in f90md directory.
%f90md/フォルダ以下に、プログラムがコピーされます。

\begin{shadebox}
    %====
   Be careful not to modify the programs directly but to make  a copy for modification. % githubからcloneしたプログラムを書換えないようにして下さい。
    %gitはあなたの書換えを尊重します。その結果、下記の方法でセクションを切り替える場合にも、
    %あなたが書換えた部分を温存するために、checkoutを拒否する場合があります。
    %書換えたい場合は、別のフォルダにさらにコピーしてそちらを変更するようにして下さい。
    %====
\end{shadebox}

If you want to get the program explained in a section, type the following command in f90md directory:
\begin{screen}
\begin{verbatim}
 git checkout sectionname
\end{verbatim}
\end{screen}

For example, if you want to get the programs of 020TwoBodyLJ section, just type
\begin{screen}
\begin{verbatim}
 git checkout 020TwoBodyLJ
\end{verbatim}
\end{screen}

If you want to see the difference between the programs of the section and those of the previous section (e.g. 010TwoBody), type the following command.
\begin{screen}
\begin{verbatim}
 git diff 010TwoBody
\end{verbatim}
\end{screen}
It shows the change history.  The lines starting from ``+'' indicates that the line is added, and ``-'' is deletion.
